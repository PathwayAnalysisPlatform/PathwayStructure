\documentclass[tikz,class=minimal,border=0pt]{standalone}

\usetikzlibrary{shapes,arrows}

\usepackage{booktabs}
\usepackage{multirow}
\usepackage{amsmath}
\usepackage{color}

\usepackage{fontspec}
\setmainfont{Arial}

\definecolor{fillcolour}{HTML}{89d0F5}

\begin{document}
\tikzstyle{elementb}=[shape = rectangle,
                     draw = fillcolour!200,
                     fill = fillcolour,
                     node distance = 50,
                     thick, rounded corners]
\tikzstyle{thingy}=[shape = rectangle,
                    draw = white,
                    draw opacity = 0,
                    fill = white,
                    fill opacity = 0,
                    text opacity = 1,
                    node distance = 20]
\tikzstyle{thingy1}=[shape = rectangle,
                    draw = white,
                    draw opacity = 0,
                    fill = white,
                    fill opacity = 0,
                    text opacity = 1,
                    node distance = -15]
\begin{tikzpicture}[scale=1]
  \node          (Middle)                             {};
  \node          (TopMiddle)    [above of=Middle]     {};
  \node          (BottomMiddle) [below of=Middle]     {};
  \node[elementb] (ProteinA)  [left of=Middle]       {Protein A};
  \node[elementb] (ProteinB) [right of=Middle]      {Protein B};
  \draw[thick,- triangle 45] (ProteinA) .. controls (TopMiddle) .. (ProteinB);
  \draw[thick,- triangle 45] (ProteinB) .. controls (BottomMiddle) .. (ProteinA);
\end{tikzpicture}

\end{document}
