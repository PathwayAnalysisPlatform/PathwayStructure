\documentclass[tikz,class=minimal,border=0pt]{standalone}

\usetikzlibrary{shapes,arrows}

\usepackage{booktabs}
\usepackage{multirow}
\usepackage{amsmath}
\usepackage{color}

\usepackage{fontspec}
\setmainfont{Arial}

\definecolor{fillcolour}{HTML}{89d0F5}

\setlength{\parindent}{0pt}

\begin{document}
\tikzstyle{elementb}=[shape = rectangle,
                     draw = fillcolour!200,
                     fill = fillcolour,
                     node distance = 50,
                     thick, rounded corners]
\tikzstyle{thingy}=[shape = rectangle,
                    draw = white,
                    draw opacity = 0,
                    fill = white,
                    fill opacity = 0,
                    text opacity = 1,
                    node distance = 60]
\tikzstyle{thingy2}=[shape = rectangle,
                    draw = white,
                    draw opacity = 0,
                    fill = white,
                    fill opacity = 0,
                    text opacity = 1,
                    node distance = 30]
\begin{tikzpicture}[scale=1]
  \node[thingy2]  (Middle)        {};
  \node[elementb] (Reaction)    [left of=Middle]  {Reaction};
  \node[elementb] (ProteinAin)  [above left of=Reaction]    {Protein A};
  \node[elementb] (ProteinBin)  [below left of=Reaction]    {Protein B};
  \node[elementb] (Complex)   [right of=Middle]   {Complex};
  \node[elementb] (ProteinAout)  [above right of=Complex]    {Protein A};
  \node[elementb] (ProteinBout)  [below right of=Complex]    {Protein B};
  % \node[thingy] (Input)     [above of=Reaction] {Input};
  % \node[thingy] (Output)    [above of=Complex]  {Output};
  \draw[thick,- triangle 45] (Reaction) -- (Complex);
  \draw[thick,- triangle 45] (ProteinAin) -- (Reaction);
  \draw[thick,- triangle 45] (ProteinBin) -- (Reaction);
  \draw[thick,-] (Complex) -- (ProteinAout);
  \draw[thick,-] (Complex) -- (ProteinBout);
\end{tikzpicture}

\end{document}

